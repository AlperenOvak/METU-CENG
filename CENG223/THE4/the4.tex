\documentclass[12pt]{article}
\usepackage[utf8]{inputenc}
\usepackage{float}
\usepackage{amsmath}
\usepackage{amssymb}
\usepackage{tikz}

\usepackage[hmargin=3cm,vmargin=6.0cm]{geometry}
%\topmargin=0cm
\topmargin=-2cm
\addtolength{\textheight}{6.5cm}
\addtolength{\textwidth}{2.0cm}
%\setlength{\leftmargin}{-5cm}
\setlength{\oddsidemargin}{0.0cm}
\setlength{\evensidemargin}{0.0cm}

%misc libraries goes here
\usepackage{fitch}

\begin{document}

\section*{Student Information } 
%Write your full name and id number between the colon and newline
%Put one empty space character after colon and before newline
Full Name :  Alperen OVAK\\
Id Number :  2580801\\

% Write your answers below the section tags
\section*{Answer 1}


\section*{Answer 2}

To find the generating function for the sequence
\[ <1, 4, 7, 10, 13, \ldots >\]
We can use the formula for an arithmetic sequence.

The \( n^{th} \) term of an arithmetic sequence can be represented as:
\[ a_n = a_1 + (n - 1) \cdot d \]
where:
\begin{align*}
a_n & \text{ is the } n^{th} \text{ term of the sequence}, \\
a_1 & \text{ is the first term of the sequence}, \\
n & \text{ is the term number (position)}, \\
d & \text{ is the common difference between consecutive terms}.
\end{align*}
From the given sequence, we can observe that:
\[ a_1 = 1 \] (first term)
and
\[ d = 3 \] (since the difference between consecutive terms is 3).

Plugging these values into the formula, we get:
\[ a_n = 1 + (n - 1) \cdot 3 \]
which simplifies to:
\[ a_n = 1 + 3n - 3 \]
and further simplifies to:
\[ a_n = 3n - 2 \]

Starting with the basic generating functions:
\begin{enumerate}
    \item The generating function for the sequence \( <1, 1, 1, 1, \dots> \) is:
    \[ \frac{1}{1-x} \]

    \item Differentiating the above generating function gives:
    \[ \frac{d}{dx} \left( \frac{1}{1-x} \right) = \frac{1}{(1-x)^2} \]
    Which represents the sequence \( <1, 2, 3, 4, \dots> \).

    \item Using the scaling theorem:
    \begin{align*}
    &\text{Multiplying } \frac{1}{1-x} \text{ by 2 gives:} \\
    &\frac{2}{1-x} \\
    &\text{Which represents the sequence } <2, 2, 2, 2, \dots>.
    \end{align*}

    \begin{align*}
    &\text{Multiplying } \frac{1}{(1-x)^2} \text{ by 3 gives:} \\
    &\frac{3}{(1-x)^2} \\
    &\text{Which represents the sequence } <3, 6, 9, 12, \dots>.
    \end{align*}

    \item Subtracting the generating function for the sequence \( <2, 2, 2, 2, \dots> \) from the generating function for the sequence \( <3, 6, 9, 12, \dots> \) gives:
    \[ \frac{3}{(1-x)^2} - \frac{2}{1-x} \]
    This resulting generating function represents the sequence \( <1, 4, 7, 10, 13, \dots> \).
\end{enumerate}

Therefore, \( \frac{1+2x}{(1-x)^2} \)  is the generating function for the sequence \( <1, 4, 7, 10, 13, \dots> \).



\section*{Answer 3}

Define the generating function for the sequence \( a_n \) as:
\[ A(x) = \sum_{n=0}^{\infty} a_n x^n \]

Multiply both sides of the recurrence relation by \( x^n \) and sum over all valid values of \( n \):
\begin{align*}
\sum_{n=1}^{\infty} a_n x^n &= \sum_{n=1}^{\infty} a_{n-1} x^n + \sum_{n=1}^{\infty} 2^n x^n \\
\end{align*}

\begin{align*}
&= x\sum_{n=1}^{\infty} a_{n-1} x^{n-1} + \sum_{n=1}^{\infty} 2^n x^n
\end{align*}

We can write \( \sum_{n=1}^{\infty} a_{n-1} x^{n-1}\) as \( \sum_{n=0}^{\infty} a_{n} x^{n} = A(x) \) \\

Furthermore, we know that  \( \sum_{n=0}^{\infty} 2^n x^n = \frac{1}{1-2x} \),\\

Thus \( \sum_{n=1}^{\infty} 2^n x^n = \frac{1}{1-2x} - 1\)\\

Let's put them into the equation we have above:

\begin{align*}
A(x)-a_{0} &= xA(x) + \frac{1}{1-2x} - 1\\
A(x)-1 &= xA(x) + \frac{1}{1-2x} - 1\\
A(x)(1-x) &= \frac{1}{1-2x}\\
\end{align*}

Therefore, \( A(x) \) is:

\begin{align*}
A(x)&= \frac{1}{(1-2x)(1-x)} = \frac{B}{(1-x)} + \frac{C}{(1-2x)}  \ \ B,C \in \mathbb{R}\\
\end{align*}

So, we have \( B=-1\) and \(C=2\)

\begin{align*}
A(x)&= \frac{1}{(1-2x)(1-x)} = \frac{-1}{(1-x)} + \frac{2}{(1-2x)}  \\
\end{align*}

Since sequence of \( \frac{1}{1-x} \) generating function is \( <1, 1, 1, 1, \dots> \),\\

Sequence of \(\frac{-1}{(1-x)}\) is \( <-1, -1, -1, -1, \dots> \).\\

Since sequence of \( \frac{1}{1-2x} \) generating function is \( <1, 2, 4, 8, \dots> \),\\

Sequence of \(\frac{2}{(1-2x)}\) is \( <2, 4, 8, 16, \dots> \).\\

Therefore sequence of \( A(x) \) is \[ <1, 3, 7, 15, \dots, 2^{n+1}-1 , \dots>\].\\
Thus, we can say that: \[ a_n = 2^{n+1}-1\]



\section*{Answer 4}


\end{document}
