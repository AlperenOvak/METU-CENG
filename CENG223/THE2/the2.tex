\documentclass[12pt]{article}
\usepackage[utf8]{inputenc}
\usepackage{float}
\usepackage{amsmath}
\usepackage{amssymb}


\usepackage[hmargin=3cm,vmargin=6.0cm]{geometry}
%\topmargin=0cm
\topmargin=-2cm
\addtolength{\textheight}{6.5cm}
\addtolength{\textwidth}{2.0cm}
%\setlength{\leftmargin}{-5cm}
\setlength{\oddsidemargin}{0.0cm}
\setlength{\evensidemargin}{0.0cm}

%misc libraries goes here
\usepackage{fitch}

\begin{document}

\section*{Student Information } 
%Write your full name and id number between the colon and newline
%Put one empty space character after colon and before newline
Full Name :  Alperen OVAK\\
Id Number :  2580801\\

% Write your answers below the section tags
\section*{Answer 1}

\paragraph{(a)}
Let \(x_1, x_2, \ldots, x_m\) be \(m\) points in \(\mathbb{C}\), and let \(\lambda_1, \lambda_2, \ldots, \lambda_m\) be non-negative coefficients summing up to 1 (\(\sum_{i=1}^{m} \lambda_i = 1\)). Consider the linear combination
\[
\sum_{i=1}^{m} \lambda_i x_i.
\]
Since \(\mathbb{C}\) is convex, any convex combination of points in \(\mathbb{C}\) is also in \(\mathbb{C}\). Therefore,
\[
\sum_{i=1}^{m} \lambda_i x_i \in \mathbb{C}.
\]

\paragraph{(b)}
Suppose \(f \circ g\) is not convex. This means there exist points \(x_1\) and \(x_2\) in the domain of \(f \circ g\) and \(t \in [0, 1]\) such that
\[
f \circ g\left((1 - t)x_1 + t x_2\right) > (1 - t) \left(f \circ g(x_1)\right) + t \left(f \circ g(x_2)\right).
\]

Now, because \(f\) is convex,
\[
f\left((1 - t)g(x_1) + tg(x_2)\right) \leq (1 - t)f(g(x_1)) + tf(g(x_2)),
\]
contradicting the assumption. Therefore, \(f \circ g\) must be convex.

\paragraph{(c)}


\section*{Answer 2}
\paragraph{(a)}
\(X\) is in this set because \(X - X = \emptyset\). \\
\\
If \(A\) is in this set, then \(X - A\) is either finite or \(\emptyset\). However, the complement of \(A\), \(X - A\), is not necessarily in this set because if \(X - A\) is finite, then \(A\) might be infinite.
\\ \\
This set is not closed under countable unions. If we take countable unions of sets where \(X - U\) is finite, the result could be a set where \(X - U\) is infinite.
\\ \\
So, this set is not a \(\sigma\)-algebra on \(X\).
\paragraph{(b)}
\paragraph{(c)}
\(X\) is in this set because \(X - X = \emptyset\).\\
\\
If \(A\) is in this set, then \(X - A\) is either infinite, \(\emptyset\), or \(X\). However, the complement of \(A\), \(X - A\), is not necessarily in this set because if \(X - A\) is infinite, then \(A\) might be finite.
\\ \\
This set is not closed under countable unions. If we take countable unions of sets where \(X - U\) is infinite, the result could be a set where \(X - U\) is finite.
\\ \\
So, this set is not a \(\sigma\)-algebra on \(X\).
\\

\section*{Answer 3}

\paragraph{(a)}
Let's consider the congruence \(ax \equiv b \pmod{p}\). If there exists an integer solution \(x=x_0\), then \(ax_0 \equiv b \pmod{p}\). So \(ax_0 - b\) is divisible by \(p\). This implies there exists \(y \in \mathbb{Z}\) such that \(ax_0 - yp = b\).
\\ \\
Let \(d = \gcd(a, p)\) (by Bezout's identity, there exist integers \(x_1\) and \(y_1\) such that \(ax_1 + py_1 = d\)). This implies: 
\\ \\
\[\left(\frac{b}{d}\right)ax_1 + \left(\frac{b}{d}\right)py_1 = b \]
\\ \\
Then, we have:
\[ ax + py = b \]
where \(x, y\) are integers. Since \(d = \gcd(a, p)\) divides both \(p\) and \(a\), it divides \(b\) too. 
\\ \\
Therefore, \(\gcd(a, p) \mid b\) is a must.

\paragraph{(b)}
\paragraph{(c)}


\section*{Answer 4}
\paragraph{(a)}
Let \(Y = \bigcup_{i \in \mathbb{N}} Y_i\).

The sets \(Y_i\) are countable; therefore, there exist surjective functions \(f_i: \mathbb{N} \rightarrow Y_i\).

By Cantor's first diagonal argument, it is known that \(\mathbb{N} \times \mathbb{N}\) is countable.

So let's define:

\[ F: \mathbb{N} \times \mathbb{N} \rightarrow Y \]

\[ (i, x) \mapsto f_i(x) \]

Per the definition of the union, this mapping is surjective.

So, \(Y\) is indeed countable.


\paragraph{(b)}
Let's denote this set by \(X^{\omega}\). Then we will show that a function \(g : \mathbb{Z}^+ \rightarrow X^{\omega}\) cannot be surjective
to prove the uncountability of this set.
\\ \\
Let's denote this set by \(X^\omega\). We will show that a function \(g:\mathbb{Z}^+ \rightarrow X^\omega\) cannot be surjective to prove the uncountability of this set.

For a function \(g\) defined as \(g(n) = (x_{n1}, x_{n2}, \ldots, x_{nn}, \ldots)\) where each \(x_{ij}\) belongs to the set \(X = \{a, b, \ldots, z\}\), consider the element \(y = (y_1, y_2, \ldots) \in X^\omega\) given by:

\[
y_n = 
\begin{cases}
    x_{nn} & \text{if } x_{nn} \neq a \\
    b & \text{if } x_{nn} = a
\end{cases}
\]
\\ \\
In other words, \(y\) is constructed such that it differs from each \(g(n)\) by at least one coordinate. This means that \(y\) is not mapped to by \(g\), and therefore, \(g\) cannot be surjective.
\\ \\
This argument generalizes to any countable product of a set \(X\) with \(\lvert X \rvert > 1\). If \(X\) has \(\lvert X \rvert = k\) elements, then there are \(k^{\mathbb{N}}\) distinct sequences in the countable product \(X^\omega\), making it uncountable.


\end{document}