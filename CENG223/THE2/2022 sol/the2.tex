\documentclass[12pt]{article}
\usepackage[utf8]{inputenc}
\usepackage{float}
\usepackage{amsmath}
\usepackage{amssymb}


\usepackage[hmargin=3cm,vmargin=6.0cm]{geometry}
%\topmargin=0cm
\topmargin=-2cm
\addtolength{\textheight}{6.5cm}
\addtolength{\textwidth}{2.0cm}
%\setlength{\leftmargin}{-5cm}
\setlength{\oddsidemargin}{0.0cm}
\setlength{\evensidemargin}{0.0cm}

%misc libraries goes here
\usepackage{fitch}

\begin{document}

\section*{Student Information } 
%Write your full name and id number between the colon and newline
%Put one empty space character after colon and before newline
Full Name :  Alperen OVAK\\
Id Number :  2580801\\

% Write your answers below the section tags
\section*{Answer 1}

\paragraph{(a)}

\subsection*{Base Case ($m = 3$):}

Assume $\mathbf{x_1, x_2, x_3}$ are three arbitrary points in the convex set $\mathcal{C}$. We want to show that any linear combination of these three points is also in $\mathcal{C}$.

Consider $\mathbf{x_1, x_2, x_3}$ and $\lambda_1, \lambda_2, \lambda_3$ such that $\mathbf{x_1} \geq 0$, $\mathbf{x_2} \geq 0$, $\mathbf{x_3} \geq 0$, $\lambda_1 \geq 0$, $\lambda_2 \geq 0$, $\lambda_3 \geq 0$, and $\mathbf{x_1} + \mathbf{x_2} + \mathbf{x_3} = 1$, $\lambda_1 + \lambda_2 + \lambda_3 = 1$. Now, consider the linear combination
\[
\lambda_1 \mathbf{x_1} + \lambda_2 \mathbf{x_2} + \lambda_3 \mathbf{x_3}.
\]

Since $\lambda_1 \mathbf{x_1} + (1 - \lambda_1) \mathbf{x_2}$ is in $\mathcal{C}$ (as given in the premise), we can use the same reasoning for the pair $(\lambda_1 \mathbf{x_1} + (1 - \lambda_1) \mathbf{x_2})$ and $\mathbf{x_3}$. Therefore, the linear combination $\lambda_3(\lambda_1 \mathbf{x_1} + (1 - \lambda_1) \mathbf{x_2}) + (1 - \lambda_3) \mathbf{x_3}$ is also in $\mathcal{C}$.

Now, let's express this in terms of $\lambda_i$'s:
\[
\begin{aligned}
&\lambda_3(\lambda_1 \mathbf{x_1} + (1 - \lambda_1) \mathbf{x_2}) + (1 - \lambda_3) \mathbf{x_3} \\
&= \lambda_3 \lambda_1 \mathbf{x_1} + \lambda_3 (1 - \lambda_1) \mathbf{x_2} + (1 - \lambda_3) \mathbf{x_3} \\
\end{aligned}
\]

Now, observe that $\lambda_3 \lambda_1 + \lambda_3 (1 - \lambda_1) + (1 - \lambda_3) = 1$, which satisfies $\sum_{i=1}^{3} \lambda_i = 1$.

Therefore, the base case $m = 3$ holds.
\subsection*{Inductive Step:}

Assume that the statement holds for $m = k$, where $k$ is some positive integer greater than $3$. That is, for any set of $k$ points $\mathbf{x_1, x_2, \ldots, x_k}$ in $\mathcal{C}$ and coefficients $\lambda_1, \lambda_2, \ldots, \lambda_k$ satisfying $\mathbf{x_i} \geq 0$, $\lambda_i \geq 0$, and $\sum_{i=1}^{k} \lambda_i = 1$, the linear combination $\sum_{i=1}^{k} \lambda_i \mathbf{x_i}$ is in $\mathcal{C}$.

Now, we want to prove the statement for $m = k + 1$. Consider $\mathbf{x_1, x_2, \ldots, x_k, x_{k+1}}$ as $k + 1$ points in $\mathcal{C}$ and coefficients $\lambda_1, \lambda_2, \ldots, \lambda_k, \lambda_{k+1}$ such that $\mathbf{x_i} \geq 0$, $\lambda_i \geq 0$, and $\sum_{i=1}^{k+1} \lambda_i = 1$.

By the inductive assumption, the linear combination $\sum_{i=1}^{k} \lambda_i \mathbf{x_i}$ is in $\mathcal{C}$. Now, consider the convex combination of this result with $\mathbf{x_{k+1}}$ using the coefficients $\lambda_{k+1}$:

\[
\lambda_{k+1} \left(\sum_{i=1}^{k} \lambda_i \mathbf{x_i}\right) + (\lambda_{k+1}-1) \mathbf{x_{k+1}}.
\]

Since $\mathcal{C}$ is convex, this convex combination is also in $\mathcal{C}$.

Therefore, by mathematical induction, the statement holds for all $m > 3$.

This completes the proof.


\paragraph{(b)}
counterexample to illustrate that the composition of convex functions is not always convex: \\

Let \( f: \mathbb{R} \rightarrow \mathbb{R} \) be defined as \( f(x) = x^2 \), which is a convex function. \\

Let \( g: \mathbb{R} \rightarrow \mathbb{R} \) be defined as \( g(x) = x^2-1 \), which is also convex. \\

Then we have \( h = f \circ g \), where \( h(x) = f(g(x)) \) be defined as \( h(x) = x^4-2x^2+1 \). \\

Let's choose two points $\mathbf{x}_1 = -1$ and $\mathbf{x}_2 = 1$, and $\mathbf{t} = 0.3$ in the range $[0, 1]$. \\
\[
    h(0.3(-1) + (0.7)1) = h(0.4) = 0.7056
\]
\[
    0.3h(-1) + (0.7)h(1) = 0
\]
\[
    0.7056 > 0
\]


Therefore \( h(x) = x^4-2x^2+1 \) is not convex, so we cannot say \( f \circ g \) is convex if \( g \) and \( f \) are convex functions.
\paragraph{(c)}

\textbf{Implication 1:}

Assume \( f(·) \) is a convex function. We should show that \( S \) is a convex set, and \( g(t) = f(x + tv) \) is convex for all \( t \) such that \( x + tv \in S \).

\begin{enumerate}
    \item \textbf{Convexity of \( S \):} Since \( f(·) \) is defined on \( S \), and \( f(·) \) is convex, it implies that \( S \) must be a convex set. This is because the domain of a convex function is always convex.
    
    \item \textbf{Convexity of \( g(t) = f(x + tv) \):} Let \( y_1 = x + t_1v \) and \( y_2 = x + t_2v \) be two points in \( S \) where \( t_1, t_2 \) are such that \( x + t_1v, x + t_2v \in S \). \\
    Now, consider \( z = \lambda y_1 + (1 - \lambda) y_2 \), where \( \lambda \) is a convex combination coefficient (\( 0 \leq \lambda \leq 1 \)). \( z = \lambda (x + t_1v) + (1 - \lambda)(x + t_2v) \) and \( z = x + (\lambda t_1 + (1 - \lambda)t_2)v \). \\
    Since \( S \) is convex, \( x + (\lambda t_1 + (1 - \lambda)t_2)v \in S \), and by the convexity of \( f(·) \), \( g(t) = f(x + tv) \) is convex.
\end{enumerate}

Therefore, the first implication holds. \\

\textbf{Implication 2:}

Assume \( S \) is a convex set, and \( g(t) = f(x + tv) \) is convex for all \( t \) such that \( x + tv \in S \). We want to show that \( f(·) \) is a convex function.

To show that \( f(·) \) is convex, we need to consider two arbitrary points \( x_1, x_2 \) in the domain of \( f(·) \) and show that \( f(\lambda x_1 + (1 - \lambda) x_2) \leq \lambda f(x_1) + (1 - \lambda) f(x_2) \) for all \( \lambda \) in \([0, 1]\).

Consider \( x_1, x_2 \) in the domain of \( f(·) \). Let \( \lambda \) be a convex combination coefficient (\( 0 \leq \lambda \leq 1 \)). Now, consider \( z = \lambda x_1 + (1 - \lambda) x_2 \). Since \( S \) is convex, \( z \) is also in \( S \). Therefore, we can use the convexity of \( g(t) = f(x + tv) \) for \( t \) such that \( x + tv = z \).
\[ g(t) = f(x + tv) = f(\lambda x_1 + (1 - \lambda) x_2) \]
By the convexity of \( g(t) \):
\[ f(\lambda x_1 + (1 - \lambda) x_2) \leq \lambda f(x_1) + (1 - \lambda) f(x_2) \]

Therefore, the second implication holds.

Since both implications hold, we can conclude that a function \( f(·) \) is convex if and only if \( S \) is a convex set, and the function \( g(t) = f(x + tv) \) is convex for all \( t \) such that \( x + tv \in S \).


\section*{Answer 2}
\paragraph{(a)}
\subparagraph{(i) if X is uncountable} 
The set of all \( U \subseteq X \) is not a \(\sigma\)-algebra on \(X\) \\ \\
\par \hspace*{1em}Let's show a counterexample: \\
\par \hspace*{1em}Let \( X = \mathbb{R} \) and \( U_1 = \mathbb{R} - \{1\} \). \\
\par \hspace*{1em}Since \( X - U_1 = \{1\} \) satisfies the condition \( U_1 \) must be in this set where \( U \subseteq X \).\\
\par \hspace*{1em}If this set is denoted by \( \Sigma \), then from property (2) \( X - U_1 = U_2 \) must be in this set.\\

\par \hspace*{1em}However, since \( X - U_2 = \mathbb{R} - \{1\} \), which is infinite, \( U_2 \) cannot be in this set. This leads to a contradiction.\\

\par \hspace*{1em}Therefore, if \( X \) is an uncountable infinite set, the set of all \( U \subseteq X \) such that \( X - U \) is either finite or is \(\emptyset\) is \textbf{not}  a \( \sigma \)-algebra on \( X \).\\

\subparagraph{(ii) if X is countable infinite}
The set of all \( U \subseteq X \) is not a \(\sigma\)-algebra on \(X\) \\ \\
\par \hspace*{1em}Let's show a counterexample: \\
\par \hspace*{1em}Let \( X = \mathbb{Z} \) and \( U_1 = \mathbb{Z} - \{1\} \). \\
\par \hspace*{1em}Since \( X - U_1 = \{1\} \) satisfies the condition \( U_1 \) must be in this set where \( U \subseteq X \).\\
\par \hspace*{1em}If this set is denoted by \( \Sigma \), then from property (2) \( X - U_1 = U_2 \) must be in this set.\\

\par \hspace*{1em}However, since \( X - U_2 = \mathbb{Z} - \{1\} \), which is infinite, \( U_2 \) cannot be in this set. This leads to a contradiction.\\

\par \hspace*{1em}Therefore, if \( X \) is an countable infinite set, the set of all \( U \subseteq X \) such that \( X - U \) is either finite or is \(\emptyset\) is \textbf{not}  a \( \sigma \)-algebra on \( X \).\\

\subparagraph{(iii) if X is finite}
\ \\
\par \hspace*{1em}The set in question must contain the empty set \( \emptyset \). This property is satisfied because \( X - X = \emptyset \), and \( \emptyset \) itself is also part of the set. \\
\par \hspace*{1em}Since every \(U\) is finite where all \( U \subseteq X \), \(X-U\) is also finite and this satisfies the condiciton. \\
\par \hspace*{1em}Therefore, if this set is denoted by \( \Sigma \), then \( X-U \) must be in this set. \\
\par \hspace*{1em}Since \( X - (X-U) = U\) is finite, this satisfies the condiciton. \\ \\

\par \hspace*{1em}Since each \( U \) where \( U \subseteq X \) satisfies the condiciton, the set of all \( U \subseteq X \) is \( P(X) \). \\
\par \hspace*{1em}Since \( \Sigma \subseteq P(X) \), Therefore, if \( X \) is an uncountable infinite set, the set of all \( U \subseteq X \) such that \( X - U \) is either finite or is \(\emptyset\) is a \( \sigma \)-algebra on \( X \).


\paragraph{(b)}
\subparagraph{(i) if X is uncountable}
The set of all \( U \subseteq X \) is not a \(\sigma\)-algebra on \(X\) \\ \\
\par \hspace*{1em}Let's show a counterexample: \\
\par \hspace*{1em}Let \( X = \mathbb{R} \) and \( U_1 = \mathbb{R} - \{1\} \). \\
\par \hspace*{1em}Since \( X - U_1 = \{1\} \) satisfies the condition \( U_1 \) must be in this set where \( U \subseteq X \).\\
\par \hspace*{1em}If this set is denoted by \( \Sigma \), then from property (2) \( X - U_1 = U_2 \) must be in this set.\\

\par \hspace*{1em}However, since \( X - U_2 = \mathbb{R} - \{1\} \), which is uncountable, \( U_2 \) cannot be in this set. This leads to a contradiction.\\

\par \hspace*{1em}Therefore, if \( X \) is an uncountable infinite set, the set of all \( U \subseteq X \) such that \( X - U \) is either finite or is \(\emptyset\) is \textbf{not}  a \( \sigma \)-algebra on \( X \).\\

\subparagraph{(ii) if X is countable infinite}
\ \\
\par \hspace*{1em}The set in question must contain \(X\). This property is satisfied because \( X - \emptyset = X \), and \( X \) itself is also part of the set. \\
\par \hspace*{1em}Since every \(U\) is countable where \( U \subseteq X \), \(X-U\) is also uncountable and this satisfies the condiciton. \\
\par \hspace*{1em}Therefore, if this set is denoted by \( \Sigma \), then \( X-U \) must be in this set. \\
\par \hspace*{1em}Since \( X - (X-U) = U\) is countable, this satisfies the condiciton. \\ \\

\par \hspace*{1em}Since each \( U \) where \( U \subseteq X \) satisfies the condiciton, the set of all \( U \subseteq X \) is \( P(X) \). \\
\par \hspace*{1em}Since \( \Sigma \subseteq P(X) \), Therefore, if \( X \) is an countable infinite set, the set of all \( U \subseteq X \) such that \( X - U \) is either countable or is all of \(X\) is a \( \sigma \)-algebra on \( X \).
\subparagraph{(iii) if X is finite}

\par \hspace*{1em}$X$ is in $\Sigma$:

\par \hspace*{1em}Since $X - X = \emptyset$ (the empty set) is finite, $X$ is in $\Sigma$.

\par \hspace*{1em}$\Sigma$ is closed under complementation:

\par \hspace*{1em}If $U$ is in $\Sigma$, then $X - U$ is finite or all of $X$. The complement of $U$ is $X - U$. If $X - U$ is finite, then $U$ is in $\Sigma$. If $X - U$ is all of $X$, then $U$ is also in $\Sigma$. Therefore, $\Sigma$ is closed under complementation.

\par \hspace*{1em}$\Sigma$ is closed under finite unions:

\par \hspace*{1em}Let $A_1, A_2, \ldots$ be sets in $\Sigma$. This means that for each $A_i$, $X - A_i$ is either finite or all of $X$.

\par \hspace*{1em}Consider the union $A = A_1 \cup A_2 \cup \ldots$. The complement of $A$ is
\[
X - A = (X - A_1) \cap (X - A_2) \cap \ldots.
\]

\par \hspace*{1em}If for each $i$, $X - A_i$ is finite, then $X - A$ is also finite (finite union of finite sets is finite). If for each $i$, $X - A_i$ is all of $X$, then $X - A$ is also all of $X$.

\par \hspace*{1em}Therefore, $\Sigma$ is closed under finite unions.

\par \hspace*{1em}Since the set $\Sigma$ satisfies all three properties, it is a $\sigma$-algebra on the finite set $X$.


\paragraph{(c)}
\subparagraph{(i) if X is uncountable}
The set of all \( U \subseteq X \) is not a \(\sigma\)-algebra on \(X\) \\ \\
\par \hspace*{1em}Let's show a counterexample: \\
\par \hspace*{1em}Let \( X = \mathbb{R} \) and \( U_1 = \{1\} \). \\
\par \hspace*{1em}Since \( X - U_1 = \mathbb{R} - \{1\} \) satisfies the condition. \( U_1 \) must be in this set where \( U \subseteq X \).\\
\par \hspace*{1em}If this set is denoted by \( \Sigma \), then from property (2) \( X - U_1 = U_2 \) must be in this set.\\

\par \hspace*{1em}However, since \( X - U_2 = \{1\} \), which is finite, \( U_2 \) cannot be in this set. This leads to a contradiction.\\

\par \hspace*{1em}Therefore, if \( X \) is an uncountable infinite set, the set of all \( U \subseteq X \) such that \( X - U \) is infinite or \(\emptyset\) or \(X\) is \textbf{not}  a \( \sigma \)-algebra on \( X \).\\

\subparagraph{(ii) if X is countable infinite}
The set of all \( U \subseteq X \) is not a \(\sigma\)-algebra on \(X\) \\ \\
\par \hspace*{1em}Let's show a counterexample: \\
\par \hspace*{1em}Let \( X = \mathbb{Z} \) and \( U_1 = \{1\} \). \\
\par \hspace*{1em}Since \( X - U_1 = \mathbb{Z} - \{1\} \) satisfies the condition. \( U_1 \) must be in this set where \( U \subseteq X \).\\
\par \hspace*{1em}If this set is denoted by \( \Sigma \), then from property (2) \( X - U_1 = U_2 \) must be in this set.\\

\par \hspace*{1em}However, since \( X - U_2 = \{1\} \), which is finite, \( U_2 \) cannot be in this set. This leads to a contradiction.\\

\par \hspace*{1em}Therefore, if \( X \) is an countable infinite set, the set of all \( U \subseteq X \) such that \( X - U \) is infinite or \(\emptyset\) or \(X\) is \textbf{not}  a \( \sigma \)-algebra on \( X \).\\

\subparagraph{(iii) if X is finite}
\par \hspace*{1em}
\par \hspace*{1em}\textbf{1. $X$ is in $\Sigma$:}

In this case, $X$ itself is in the set because $X-X=$ \(\emptyset\) is in the set. So, the first property is satisfied.

\par \hspace*{1em}\textbf{2. $\Sigma$ is closed under complementation:}

If $A$ is in $\Sigma$, then $X - A$ is infinite or $\emptyset$ or $X$. Let's consider each case:
\begin{itemize}
    \item If $A$ is finite where $A\neq X$ and  $A\neq \emptyset$ , then $X - A$ is also finite, which is not in the set.
    \item If $A$ is $\emptyset$, then $X - A$ is $X$, which is in the set.
    \item If $A$ is $X$, then $X - A$ is $\emptyset$, which is in the set.
\end{itemize}
So, the set is closed under complementation.

\par \hspace*{1em}\textbf{3. $\Sigma$ is closed under countable unions:}

Let $A_1, A_2, \ldots$ be sets in $\Sigma$. We want to show that their union,
\[
A = A_1 \cup A_2 \cup \ldots,
\]
is also in $\Sigma$.
\begin{itemize}
    \item Since we have only two elements in this set, which is $\emptyset$ and $X$, then $A = X \cup \emptyset = X$ is also in the set.    
\end{itemize}
Since all three properties are satisfied, the given set is a $\sigma$-algebra on the finite set $X$.


\section*{Answer 3}

\paragraph{(a)}
\textbf{Suppose $ax \equiv b \pmod{p}$ has a solution.}

This implies that there exists an integer $x$ such that $ax - b = p \cdot q$ for some integer $q$.

Take $d = \gcd(a, p)$. We have:

\[
a = a \cdot t + p \cdot r
\]

where $t$ and $r$ are integers.

Since $d = \gcd(a, p)$, it follows that $d$ divides both $a$ and $p$. Therefore, we can express $a$ and $n$ as:

\[
a = d \cdot q_1 \quad \text{where} \quad q_1 \in \mathbb{Z}
\]

\[
p = d \cdot q_2 \quad \text{where} \quad q_2 \in \mathbb{Z}
\]

Lastly, from the earlier expression $ax - b = p \cdot q$, we can substitute $a$ and $n$ using the above expressions:

\[
b = a \cdot x - p \cdot q = (d \cdot q_1) \cdot x - (d \cdot q_2) \cdot q = d \cdot (q_1 \cdot x - q_2 \cdot q)
\]

Let $c = q_1 \cdot x - q_2 \cdot q$, then $b = d \cdot c$.

Thus, $d = \gcd(a, p)$ divides $b$, indicating that if there is a solution to $ax \equiv b \pmod{n}$, then $d = \gcd(a, p)$ divides $b$. \\

\textbf{Suppose $d = \gcd(a, p)$ and $d \mid b$.}

Then we have
\[
\begin{cases}
    d = a \cdot t + p \cdot r, & \text{$t, r \in \mathbb{Z}$} \\
    d \mid b
\end{cases}
\]

From these equations, it follows that
\[
b = d \cdot c = (a \cdot t + p \cdot r) \cdot c = atc + prc \implies b - a \cdot (tc) = p \cdot (rc) \implies a \cdot (tc) = b \pmod{p}
\]

This implies that $ax \equiv b \pmod{p}$ has a solution, where $x = t \cdot c$. \\

Therefore,the congruence $ax \equiv b \pmod{p}$ has a solution for $x$ if and only if $\gcd(a, p) \mid b$.
\paragraph{(b)}

To prove that the pair of congruences has a solution for \(x\) under the given conditions, we will use Bézout's identity.

Let \(d = \gcd(p_1, p_2) = 1\) (since \(\gcd(p_1, p_2) = 1\)).

By Bézout's identity, there exist integers \(s\) and \(t\) such that \(sp_1 + tp_2 = 1\).

Now, consider the system of congruences:

\begin{enumerate}
  \item \(a_1x \equiv b_1 \pmod{p_1}\)
  \item \(a_2x \equiv b_2 \pmod{p_2}\)
\end{enumerate}

We want to find a solution \(x\) that satisfies both congruences.

Let's define \(x_0 = t \cdot p_2\) (Note that \(x_0\) is a solution to the second congruence since \(a_2x_0 = a_2(t \cdot p_2) \equiv b_2 \pmod{p_2}\)).

Now, let's consider the first congruence with this value of \(x_0\):

\[a_1x_0 = a_1(t \cdot p_2) \equiv b_1 \pmod{p_1}\]

Since \(sp_1 + tp_2 = 1\), we can multiply the entire congruence by \(s\) (which is congruent to 1 modulo \(p_1\)):

\[a_1(t \cdot p_2) \cdot s \equiv b_1 \cdot s \pmod{p_1}\]

This simplifies to:

\[a_1(t \cdot p_2 \cdot s) \equiv b_1 \cdot s \pmod{p_1}\]

Since \(t \cdot p_2 \cdot s\) is congruent to 1 modulo \(p_1\), we have:

\[a_1 \cdot 1 \equiv b_1 \cdot s \pmod{p_1}\]

This implies that \(a_1 \equiv b_1 \cdot s \pmod{p_1}\). Since \(s\) is an integer, \(b_1 \cdot s\) is divisible by \(\gcd(a_1, p_1)\).

Therefore, \(a_1 \equiv b_1 \cdot s \pmod{p_1}\) implies that \(a_1x_0 \equiv b_1 \pmod{p_1}\), and \(x_0\) is a solution to the first congruence as well.

So, \(x = x_0\) is a solution to the pair of congruences.


\paragraph{(c)}

\textbf{Given:}
We consider the system of congruences:
\[
a_jx \equiv b_j \pmod{p_j}
\]
where $(a_j, p_j) = 1$ for all $j$.

\textbf{Claim (1):}
There always exists a solution for
\[
a_jx \equiv b_j \pmod{p_j}
\]
regardless of the choice of $b_j$.

\textit{Proof:}
Let $C_j$ be a solution for $a_jx \equiv b_j \pmod{p_j}$ for $j = 1,2,...,k$. Since $(a_j, p_j) = 1$, we have $[p_1, p_2, ..., p_k] = \Pi$, where $\Pi$ is the product of all $p_j$ since they are coprime. Consider $m_j$ such that $m_j\Pi \equiv 1 \pmod{p_j}$ [Equation (2)]. This gives a unique solution $x \equiv m'_j \pmod{p_j}$.

Considering
\[
x_0 = c_1m_1m'_1 + c_2m_2m'_2 + ... + c_km_km'_k,
\]
where $c_j$ is a chosen solution for $a_jx \equiv b_j \pmod{p_j}$ [Equation (1)], we can show that $x_0$ is a common solution to the system.

For $i \neq j$, $p_i$ divides $m_j = p_1p_2...p_kp_j$. Therefore,
\[
a_jx_0 \equiv \sum_{i=1}^{k} a_ic_im_im'_i \equiv a_jc_jm_jm'_j \equiv a_jc_j \pmod{p_j}
\]

Since $m_jm'_j \equiv 1 \pmod{p_j}$, we have $a_jx_0 \equiv a_jc_j \equiv b_j \pmod{p_j}$. Thus, $x_0$ is a common solution for $j = 1,2,...,k$.

\textbf{Claim (2):}
If $x$ is another solution to the system of congruences, then
\[
x \equiv x_0 \pmod{[p_1, p_2, ..., p_k]}
\].

\textit{Proof:}
Let $x$ be another solution. This implies
\[
x_0 \equiv c_j \equiv x \pmod{p_j}
\]
for all $j = 1,2,...,k$. Thus, $x_0 - x$ is a common multiple of $p_1, p_2, ..., p_k$, and hence a multiple of $[p_1, p_2, ..., p_k] = \Pi$. Therefore,
\[
x \equiv x_0 \pmod{[\Pi]}
\].

\section*{Answer 4}
\paragraph{(a)}
Let's denote this set by \(X^{\omega}\). Then we will show that a function \(g : \mathbb{Z}^+ \rightarrow X^{\omega}\) cannot be surjective
to prove the uncountability of this set.
\\ \\
Let's denote this set by \(X^\omega\). We will show that a function \(g:\mathbb{Z}^+ \rightarrow X^\omega\) cannot be surjective to prove the uncountability of this set.

For a function \(g\) defined as \(g(n) = (x_{n1}, x_{n2}, \ldots, x_{nn}, \ldots)\) where each \(x_{ij}\) belongs to the set \(X = \{a, b, \ldots, z\}\), consider the element \(y = (y_1, y_2, \ldots) \in X^\omega\) given by:

\[
y_n = 
\begin{cases}
    a & \text{if } x_{nn} \neq a \\
    b & \text{if } x_{nn} = a
\end{cases}
\]
\\ \\
In other words, \(y\) is constructed such that it differs from each \(g(n)\) by at least one coordinate. This means that \(y\) is not mapped to by \(g\), and therefore, \(g\) cannot be surjective.
\\ \\
This argument generalizes to any countable product of a set \(X\) with \(\lvert X \rvert > 1\). If \(X\) has \(\lvert X \rvert = k\) elements, then there are \(k^{\mathbb{N}}\) distinct sequences in the countable product \(X^\omega\), making it uncountable.



\paragraph{(b)}
Let \(Y = \bigcup_{i \in \mathbb{N}} Y_i\).

The sets \(Y_i\) are infinitely countable; therefore, there exist surjective functions \(f_i: \mathbb{N} \rightarrow Y_i\).

By Cantor's first diagonal argument, it is known that \(\mathbb{N} \times \mathbb{N}\) is countable.

So let's define:

\[ F: \mathbb{N} \times \mathbb{N} \rightarrow Y \]

\[ (i, x) \mapsto f_i(x) \]

Per the definition of the union, this mapping is surjective.

So, \(Y\) is indeed countable.

\end{document}