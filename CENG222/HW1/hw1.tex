\documentclass[12pt]{article}
\usepackage[utf8]{inputenc}
\usepackage{float}
\usepackage{amsmath}

\usepackage[hmargin=3cm,vmargin=6.0cm]{geometry}
%\topmargin=0cm
\topmargin=-2cm
\addtolength{\textheight}{6.5cm}
\addtolength{\textwidth}{2.0cm}
%\setlength{\leftmargin}{-5cm}
\setlength{\oddsidemargin}{0.0cm}
\setlength{\evensidemargin}{0.0cm}

%misc libraries goes here

\begin{document}

\section*{Student Information } 
%Write your full name and id number between the colon and newline
%Put one empty space character after colon and before newline
Full Name :  Alperen OVAK\\
Id Number :  2580801\\

% Write your answers below the section tags
\section*{Answer 1}

\subsection*{a)}
To determine the constant $N$ so that $P$ qualifies as a probability mass function, we need to ensure that the sum of all probabilities equals 1:

\[
\sum_{x=1}^{5} P(x) = 1
\]

Given that $P(x) = \frac{N}{x}$, the sum becomes:

\[
\frac{N}{1} + \frac{N}{2} + \frac{N}{3} + \frac{N}{4} + \frac{N}{5} = 1
\]

Now solve for $N$:

\[
N + \frac{N}{2} + \frac{N}{3} + \frac{N}{4} + \frac{N}{5} = 1
\]

\[
\frac{60N + 30N + 20N + 15N + 12N}{60} = 1
\]

\[
\frac{137N}{60} = 1
\]

\[
N = \frac{60}{137} \approx 0.438
\]

So, the constant $N$ is $\frac{60}{137}$ $\approx$ 0.438.

\subsection*{b)}
The expected value (mean) of a discrete random variable $X$ is given by:

\[
E(X) = \sum_{x} x \cdot P(x)
\]

Using the given probability mass function, we can calculate:

\[
E(X) = \sum_{x=1}^{5} x \cdot \frac{N}{x} = \sum_{x=1}^{5} N
\]

\[
E(X) = N + N + N + N + N = 5N
\]

Substitute $N = \frac{60}{137} $:

\[
E(X) = 5 \cdot \frac{60}{137} \approx 2.190
\]

\subsection*{c)}
The variance of a discrete random variable $X$ is given by:

\[
\text{Var}(X) = E(X^2) - [E(X)]^2
\]

First, calculate $E(X^2)$:

\[
E(X^2) = \sum_{x} x^2 \cdot P(x)
\]

\[
E(X^2) = \sum_{x=1}^{5} x^2 \cdot \frac{N}{x}
\]

\[
E(X^2) = \sum_{x=1}^{5} x \cdot N
\]

\[
E(X^2) = N + 2N + 3N + 4N + 5N = 15N
\]

Now, calculate the variance:

\[
\text{Var}(X) = 15N - (5N)^2
\]

Substitute $N = \frac{60}{137}$, we get:

\[
\text{Var}(X) \approx 1.774
\]

\subsection*{d)}
For the joint distribution, $P(x, y) = P(x)P(y)$. Calculate $E(XY)$, $E(X)$, and $E(Y)$ to find the covariance:

\[
E(XY) = \sum_{x}\sum_{y} xy \cdot P(x, y)
\]

\[
E(XY) = \sum_{x=1}^{5}\sum_{y=1}^{5} xy \cdot P(x)P(y)
\]

\[
E(XY) = \sum_{x=1}^{5}\sum_{y=1}^{5} xy \cdot \frac{N}{x} \cdot \frac{y}{15}
\]

\[
E(XY) =  \sum_{x=1}^{5}\sum_{y=1}^{5} \frac{N}{15} \cdot y^2
\]


\[
E(XY) \approx 8.029
\]

Note that $E(X)E(Y) =8.030 $. Since $E(X)E(Y) \approx E(XY)$, we can say that X and Y is independent. \\

Now, calculate $E(X)$ and $E(Y)$ using the results from parts b) and c). Finally, use the formula $\text{Cov}(X, Y) = E(XY) - E(X)E(Y)$ to find the covariance. Interpret the covariance value in terms of the relationship between $X$ and $Y$. \\

$\text{Cov}(X, Y) = E(XY) - E(X)E(Y) = 0$


\section*{Answer 2}

\subsection*{a)} 

Let $p$ be the probability of success in a single attempt. The probability of failure in a single attempt is $1 - p$.

The probability of not having a success in a single attempt is $1 - p$. The probability of not having a success in all 1000 attempts is $(1 - p)^{1000}$.

Now, the probability of at least one success in 1000 attempts is the complement of the probability of no success in all attempts:

\[
P\{X \geq 1\} = 1 - P\{X \leq 1\}
\]

Given that the probability of no success in a single attempt is $(1 - p)$, the probability of no success in all 1000 attempts is $(1 - p)^{1000}$.

So, we want to find $p$ such that:

\[
1 - (1 - p)^{1000} = 0.95
\]

Now, we can solve this equation for $p$:

\[
(1 - p)^{1000} = 0.05
\]

Taking the 1000th root of both sides:

\[
1 - p = 0.05^{1/1000}
\]

Now, solve for $p$:

\[
p = 1 - 0.05^{1/1000}
\]

Calculating this gives the probability of success for an individual attempt:

\[
p \approx 0.003
\]

So, the probability of success for an individual attempt should be approximately $0.003$ to ensure that the probability of at least one success in 1000 trials is 95%.


\subsection*{b)} 

\subsection*{c)} 


\section*{Answer 3}
\subsection*{a)} 
\subsection*{b)} 
\subsection*{c)} 


\end{document}
