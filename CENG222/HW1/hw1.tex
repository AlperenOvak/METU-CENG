\documentclass[12pt]{article}
\usepackage[utf8]{inputenc}
\usepackage{float}
\usepackage{amsmath}
\usepackage{graphicx}

\usepackage[hmargin=3cm,vmargin=6.0cm]{geometry}
%\topmargin=0cm
\topmargin=-2cm
\addtolength{\textheight}{6.5cm}
\addtolength{\textwidth}{2.0cm}
%\setlength{\leftmargin}{-5cm}
\setlength{\oddsidemargin}{0.0cm}
\setlength{\evensidemargin}{0.0cm}

%misc libraries goes here

\begin{document}

\section*{Student Information } 
%Write your full name and id number between the colon and newline
%Put one empty space character after colon and before newline
Full Name :  Alperen OVAK\\
Id Number :  2580801\\

% Write your answers below the section tags
\section*{Answer 1}

\subsection*{a)}
To determine the constant $N$ so that $P$ qualifies as a probability mass function, we need to ensure that the sum of all probabilities equals 1:

\[
\sum_{x=1}^{5} P(x) = 1
\]

Given that $P(x) = \frac{N}{x}$, the sum becomes:

\[
\frac{N}{1} + \frac{N}{2} + \frac{N}{3} + \frac{N}{4} + \frac{N}{5} = 1
\]

Now solve for $N$:

\[
N + \frac{N}{2} + \frac{N}{3} + \frac{N}{4} + \frac{N}{5} = 1
\]

\[
\frac{60N + 30N + 20N + 15N + 12N}{60} = 1
\]

\[
\frac{137N}{60} = 1
\]

\[
N = \frac{60}{137} \approx 0.438
\]

So, the constant $N$ is $\frac{60}{137}$ $\approx$ 0.438.

\subsection*{b)}
The expected value (mean) of a discrete random variable $X$ is given by:

\[
E(X) = \sum_{x} x \cdot P(x)
\]

Using the given probability mass function, we can calculate:

\[
E(X) = \sum_{x=1}^{5} x \cdot \frac{N}{x} = \sum_{x=1}^{5} N
\]

\[
E(X) = N + N + N + N + N = 5N
\]

Substitute $N = \frac{60}{137} $:

\[
E(X) = 5 \cdot \frac{60}{137} \approx 2.190
\]

\subsection*{c)}
The variance of a discrete random variable $X$ is given by:

\[
\text{Var}(X) = E(X^2) - [E(X)]^2
\]

First, calculate $E(X^2)$:

\[
E(X^2) = \sum_{x} x^2 \cdot P(x)
\]

\[
E(X^2) = \sum_{x=1}^{5} x^2 \cdot \frac{N}{x}
\]

\[
E(X^2) = \sum_{x=1}^{5} x \cdot N
\]

\[
E(X^2) = N + 2N + 3N + 4N + 5N = 15N
\]

Now, calculate the variance:

\[
\text{Var}(X) = 15N - (5N)^2
\]

Substitute $N = \frac{60}{137}$, we get:

\[
\text{Var}(X) \approx 1.774
\]

\subsection*{d)}
For the joint distribution, $P(x, y) = P(x)P(y)$. Calculate $E(XY)$, $E(X)$, and $E(Y)$ to find the covariance:

\[
E(XY) = \sum_{x}\sum_{y} xy \cdot P(x, y)
\]

\[
E(XY) = \sum_{x=1}^{5}\sum_{y=1}^{5} xy \cdot P(x)P(y)
\]

\[
E(XY) = \sum_{x=1}^{5}\sum_{y=1}^{5} xy \cdot \frac{N}{x} \cdot \frac{y}{15}
\]

\[
E(XY) =  \sum_{x=1}^{5}\sum_{y=1}^{5} \frac{N}{15} \cdot y^2
\]


\[
E(XY) \approx 8.029
\]

Note that $E(X)E(Y) =8.030 $. Since $E(X)E(Y) \approx E(XY)$, we can say that X and Y is independent. \\

Now, calculate $E(X)$ and $E(Y)$ using the results from parts b) and c). Finally, use the formula $\text{Cov}(X, Y) = E(XY) - E(X)E(Y)$ to find the covariance. Interpret the covariance value in terms of the relationship between $X$ and $Y$. \\

$\text{Cov}(X, Y) = E(XY) - E(X)E(Y) = 0$


\section*{Answer 2}

\subsection*{a)} 

Let $p$ be the probability of success in a single attempt. The probability of failure in a single attempt is $1 - p$.

The probability of not having a success in a single attempt is $1 - p$. The probability of not having a success in all 1000 attempts is $(1 - p)^{1000}$.

Now, the probability of at least one success in 1000 attempts is the complement of the probability of no success in all attempts:

\[
P\{X \geq 1\} = 1 - P\{X \leq 1\}
\]

Given that the probability of no success in a single attempt is $(1 - p)$, the probability of no success in all 1000 attempts is $(1 - p)^{1000}$.

So, we want to find $p$ such that:

\[
1 - (1 - p)^{1000} = 0.95
\]

Now, we can solve this equation for $p$:

\[
(1 - p)^{1000} = 0.05
\]

Taking the 1000th root of both sides:

\[
1 - p = 0.05^{1/1000}
\]

Now, solve for $p$:

\[
p = 1 - 0.05^{1/1000}
\]

Calculating this gives the probability of success for an individual attempt:

\[
p \approx 0.003
\]

So, the probability of success for an individual attempt should be approximately $0.003$ to ensure that the probability of at least one success in 1000 trials is 95%.


\subsection*{b)} 
\textbf{i)}
Let \( X \) be the number of games played until 2 matches are won against
an IM for an average player. Since \( X \) has Negative Binomial distribution with \( k = 2\) and \( p = 0.003 \), it is a number of trials needed to see 2 won. We need \( P\{X > 500\} = \sum_{n=501}^{\infty} P(x) \) or \( 1 - F(500) \); however, there is no table of Negative Binomial distribution in the Appendix, and applying the formula for \( P(x) \) directly. \\
Instead, we can solve this problem by a Binomial distribution. Although \( X \) is not Binomial at all, the probability \( P\{X > 500\} \) can be related to some Binomial variable.

\begin{align*}
P\{X > 500\} &= P\{ \text{more than 500 games needed to get 2 won} \} \\
            &= P\{ \text{500 games are not sufficient} \} \\
            &= P\{ \text{there are fewer than 2 wons in 500 games} \} \\
            &= P\{Y < 2\} ,
\end{align*}
where \( Y \) is the number of successes (non-defective components) in 500 games, which is a Binomial variable with parameters \( n = 500 \) and \( p = 0.003 \).
\[ P\{X > 500\} = P\{Y < 2\} = P\{Y \leq 1\} = F(2) \]

We can calculate this as follows:

\[ F(2) = P(0) + P(1) \]

According to the Binomial distribution formula $P(x) = P\{X = x\} = \binom{n}{x} p^x q^{n-x}$:

\begin{align*}
P(0) &=   \binom{500}{0} p^0 q^{500-0}\\
     &=   1 \times (0.003)^0 \times (1-0.003)^{500-0} \\
     &=   1 \times 1 \times (0.997)^{500} \\
     &=   (0.997)^{500} \\
     &\approx   0.223\\
\end{align*}

\begin{align*}
P(1) &=   \binom{500}{1} p^1 q^{500-1}\\
     &=   500 \times (0.003)^1 \times (1-0.003)^{500-1} \\
     &=   500 \times (0.003) \times (0.997)^{499} \\
     &\approx   0.335\\
\end{align*}

Therefore, Our answer is:

\[ F(2) \approx 0.223 + 0.335 \approx 0.558\]

\textbf{ii)} This question is very similar to above question. We can solve it in same way:

\begin{align*}
P\{X > 10^4\} &= P\{ \text{more than $10^4$ games needed to get 2 won} \} \\
            &= P\{ \text{$10^4$ games are not sufficient} \} \\
            &= P\{ \text{there are fewer than 2 wons in $10^4$ games} \} \\
            &= P\{Y < 2\} ,
\end{align*}

Solve by Binomial variable with parameters \( n = 10^4 \) and \( p = 10^{-4} \).
\[ P\{X > 10^4\} = P\{Y < 2\} = P\{Y \leq 1\} = F(2) \]

\[ F(2) = P(0) + P(1) \]

\begin{align*}
P(0) &=   \binom{10^4}{0} p^0 q^{10^4-0}\\
     &=   1 \times (10^{-4})^0 \times (1-10^{-4})^{10^4-0} \\
     &=   1 \times 1 \times (1-10^{-4})^{10^4} \\
     &=   (1-10^{-4})^{10^4} \\
     &=   0.367861046\\
\end{align*}

\begin{align*}
P(1) &=   \binom{10^4}{1} p^1 q^{10^4-1}\\
     &=   10^4 \times (10^{-4})^1 \times (1-10^{-4})^{10^4-1} \\
     &=   0.367897836\\
\end{align*}

Therefore, Our answer is:

\[ F(2) \approx 0.367861046 + 0.367897836 \approx 0.7358\]

\subsection*{c)} 

Let \( X \) be the number of days feeled healthy. It is the number of successes in 366 Bernoulli trials, thus \( X \) is Binomial with \( n = 366 \) and \( p = 0.98 \). Poisson approximation cannot be applied to \( X \) because \( p \) is too large. However, the number of failures \( Y \) is also Binomial, with parameters \( n = 366 \) and \( q = 0.02 \), and it is approximately Poisson with \( \lambda = nq = 7.32 \). \\

From Table A3, there is no value for $\lambda =7.32 $. Thus, we need to choose most fitting values from Table A3 which is $\lambda =7.5 $
\[ P \{X \geq 360\} = P \{Y \leq 6\} = F_Y(6) \approx 0.378. \]


\section*{Answer 3}
\subsection*{a)} 
\begin{figure}[h]
    \centering
    \includegraphics{3a_kod}
    \caption{CDF of the Poisson distribution for different values of lambda with the same number of events}
\end{figure}

The Poisson CDF represents the probability that the number of events in a Poisson process will be less than or equal to a certain value. So, in this case, we're comparing the probabilities of observing 6 or fewer events in two Poisson distributions with different lambda values.

In the Poisson distribution, the parameter $\lambda$ represents the average rate of occurrence of events in a fixed interval. In this case, we're comparing two scenarios where the average rate of occurrence is slightly different: $\lambda_1 = 7.32$ and $\lambda_2 = 7.5$.

To compare the two cumulative probabilities:

- $\text{poisscdf}(6, 7.32)$: This represents the probability of observing 6 or fewer events in a Poisson process with an average rate of occurrence of $\lambda_1 = 7.32$.

- $\text{poisscdf}(6, 7.5)$: This represents the probability of observing 6 or fewer events in a Poisson process with an average rate of occurrence of $\lambda_2 = 7.5$.

Since the average rate of occurrence $\lambda_2 = 7.5$ is slightly higher than $\lambda_1 = 7.32$, it implies that there's a slightly higher chance of observing more events in the process with $\lambda_2$ compared to $\lambda_1$. Therefore, the cumulative probability of observing 6 or fewer events in the process with $\lambda_2$ ($\text{poisscdf}(6, 7.5)$) will be lower than the cumulative probability in the process with $\lambda_1$ ($\text{poisscdf}(6, 7.32)$).

In essence, the higher the average rate of occurrence ($\lambda$), the more spread out the Poisson distribution becomes, resulting in a higher probability of observing larger numbers of events, which is reflected in the cumulative distribution function.


\subsection*{b)} 

\begin{figure}[h]
    \centering
    \includegraphics[scale=0.7]{3b_kod}
    \caption{The kod of Poisson and Binomial probabilities}
\end{figure}

\begin{figure}[h]
    \centering
    \includegraphics[scale=0.5]{3b_graph}
    \caption{The graph of Poisson and Binomial probabilities}
\end{figure}

\subsection*{c)} 


\end{document}
